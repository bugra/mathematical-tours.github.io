 
\chapter{Programs \Maple{}}
\label{appendix-software-resources}
 
\index{Maple@\Maple{}} This chapter brings together all the \Maple{} programs in the book. Each program is a separate \texttt{.dsw} file. They have often been cut into several pieces for the sake of clarity.

% ------------------------------------------------- -----
% ------------------------------------------------- -----
% ------------------------------------------------- -----
% section - Transform over a finite field                           
% ------------------------------------------------- -----
% ------------------------------------------------- -----
% ------------------------------------------------- -----
\section{Transform over a finite field}
% \addcontentsline{toc}{section}{Transform over a finite field}
\label{sect1-listing-transform-finite-body}
 
\index{Finite body} The file \texttt{fft-finite-body.msw} successively realizes \begin{enumerate}
\item a search for irreducible factors of $ X^n-1 $ over a finite field $ \FF_p $ (we have taken $ p = 2 $). With the command \texttt{alias}, we name $ \alpha $ a primitive root of the unit.
\item a naive implementation of the Fourier transform on the cyclotomic field $ \FF_{p^r} $. In the case where $ n $ is of the form $ 2^s $, it is possible to implement a recursive version of the algorithm. This is done for the Fourier transform on a ring, appendix \ref{sect1-listing-transform-ring}.
\item a test on a randomly drawn $ f \in \FF_p^n $ vector. We can see that $ \wh{f} \notin \FF_p^n $, since we have to do the calculations in a cyclotomic extension of $ \FF_p $.
\end{enumerate}
\begin{listing} \begin{footnotesize}
\begin{maplegroup}
\begin{flushleft}
Les paramètres pour faire un TFD de taille 
\mapleinline{inert}{2d}{n;}{%
$n$%
} fixé sur 
\mapleinline{inert}{2d}{Fp;}{%
$\FF_p$%
} :
\end{flushleft}

\end{maplegroup}
\begin{maplegroup}
\begin{mapleinput}
\mapleinline{active}{1d}{\pwith{}(numtheory): n := 16: p := 3:}{%
}
\end{mapleinput}

\end{maplegroup}
\begin{maplegroup}
\begin{flushleft}
Liste des facteurs de 
\mapleinline{inert}{2d}{X^n-1;}{%
$X^{n} - 1$%
}. Choix d'un facteur irréductible de degré 
\mapleinline{inert}{2d}{r;}{%
$r$%
} et de la racine primive associée :
on constate que 
\mapleinline{inert}{2d}{r;}{%
$r$%
} est bien l'ordre de 
\mapleinline{inert}{2d}{p;}{%
$p$%
} dans $(\ZZ/n\ZZ)^*$.
\end{flushleft}

\end{maplegroup}
\begin{maplegroup}
\begin{mapleinput}
\mapleinline{active}{1d}{liste_div := op(Factor( cyclotomic(n,X) ) mod p );
P := liste_div[1];
alias( alpha = RootOf(P) ):}{%
}
\end{mapleinput}

\mapleresult
\begin{maplelatex}
\mapleinline{inert}{2d}{liste_div := X^4+2*X^2+2, X^4+X^2+2;}{%
\[
\mathit{liste\_div} := X^{4} + 2\,X^{2} + 2, \,X^{4} + X^{2} + 2
\]
%
}
\end{maplelatex}

\begin{maplelatex}
\mapleinline{inert}{2d}{P := X^4+2*X^2+2;}{%
\[
P := X^{4} + 2\,X^{2} + 2
\]
%
}
\end{maplelatex}

\end{maplegroup}
\begin{maplegroup}
\begin{flushleft}
Transformée de Fourier, version 
\mapleinline{inert}{2d}{O(n^2);}{%
$\mathrm{O}(n^{2})$%
} :
\end{flushleft}

\end{maplegroup}
\begin{maplegroup}
\begin{mapleinput}
\mapleinline{active}{1d}{TFD := \pproc{}(f, signe)
\quad \plocal res, formule;
\quad # pour plus de lisibilité, on écrit à part la formule de TFD :
\quad formule := 'f[l+1]*alpha^(-signe*(k-1)*l)';
\quad res := [ seq( sum( formule, 'l'=0..n-1 ) mod p , k=1..n)  ];     
\quad \pif signe=-1 \pthen res := 1/n*res mod p \pend \pif;
\quad \preturn{}(Normal(res) mod p);
\pend \pproc{}:}{%
}
\end{mapleinput}

\end{maplegroup}
\begin{maplegroup}
\begin{flushleft}
Test simple :
\end{flushleft}

\end{maplegroup}
\begin{maplegroup}
\begin{mapleinput}
\mapleinline{active}{1d}{hasard := rand(0..(p-1)):
x := [seq( hasard(), i=1..n )];
y := TFD(x,1);  # TFD(x) n'est plus à coefficients dans F_2.
evalb( x = TFD(y,-1) ); # Mais on retombe bien sur nos pattes.}{%
}
\end{mapleinput}

\mapleresult
\begin{maplelatex}
\mapleinline{inert}{2d}{x := [0, 2, 0, 2, 1, 2, 2, 2, 1, 1, 1, 0, 0, 2, 2, 1];}{%
\[
x := [0, \,2, \,0, \,2, \,1, \,2, \,2, \,2, \,1, \,1, \,1, \,0, 
\,0, \,2, \,2, \,1]
\]
%
}
\end{maplelatex}

\begin{maplelatex}
\mapleinline{inert}{2d}{y := [1, 2*alpha^3+2*alpha+alpha^2+2, 1, 2*alpha^3+alpha+2*alpha^2,
alpha^2+1, alpha^3+alpha, 1, 2*alpha^3+alpha, 1,
alpha^3+alpha+alpha^2+2, 2, alpha^3+2*alpha+2*alpha^2, 2*alpha^2+2,
2*alpha^3+2*alpha, 2, alpha^3+2*alpha];}{%
\maplemultiline{
y := [1, \,2\,\alpha ^{3} + 2\,\alpha  + \alpha ^{2} + 2, \,1, \,
2\,\alpha ^{3} + \alpha  + 2\,\alpha ^{2}, \,\alpha ^{2} + 1, \,
\alpha ^{3} + \alpha , \,1, \,2\,\alpha ^{3} + \alpha , \,1,  \\
\alpha ^{3} + \alpha  + \alpha ^{2} + 2, \,2, \,\alpha ^{3} + 2\,
\alpha  + 2\,\alpha ^{2}, \,2\,\alpha ^{2} + 2, \,2\,\alpha ^{3}
 + 2\,\alpha , \,2, \,\alpha ^{3} + 2\,\alpha ] }
%
}
\end{maplelatex}

\begin{maplelatex}
\mapleinline{inert}{2d}{true;}{%
\[
\mathit{true}
\]
%
}
\end{maplelatex}

\end{maplegroup}

% \vskip -3mm
\end{footnotesize}
 
\caption{File \texttt{fft-finite-body.msw}}
 
\label{listing-fft-finite-body}
\end{listing}
 
% ------------------------------------------------- -----
% ------------------------------------------------- -----
% ------------------------------------------------- -----
% section - Transform on a ring                           
% ------------------------------------------------- -----
% ------------------------------------------------- -----
% ------------------------------------------------- -----
\section{Transform on a ring}
% \addcontentsline{toc}{section}{Transform on a ring}
\label{sect1-listing-transform-ring}
 
 
\index{Ring} The \Maple{} \texttt{fft-ring.msw} program calculates a transform of size $ n $ to value in a ring $ \ZZ/m \ZZ $ for a judiciously chosen integer $ m $ ( in accordance with the explanations given in Paragraph~\ref{sect1-calculations-ring}). We chose $ n $ of the form $ 2^s $, which allows to implement a recursive algorithm of type FFT. We use an intermediate function, \texttt{FFT \_rec}, which allows to update the main root of the unit with each call.
 
 
 
\begin{listing} \begin{footnotesize}
\begin{maplegroup}
\begin{flushleft}
Définition des paramètres de la transformée
\end{flushleft}

\end{maplegroup}
\begin{maplegroup}
\begin{mapleinput}
\mapleinline{active}{1d}{s := 4: n := 2^s: m := 2^(2^(s-1)) + 1:}{%
}
\end{mapleinput}

\end{maplegroup}
\begin{maplegroup}
\begin{flushleft}
Sous-procédure récursive :
\end{flushleft}

\end{maplegroup}
\begin{maplegroup}
\begin{mapleinput}
\mapleinline{active}{1d}{FFT_rec := \pproc{}(f, signe, zeta)
\quad \plocal nn, n1, s, t, r;
\quad nn := nops(f); n1 := nn/2; # taille du vecteur
\quad \pif nn=1 \pthen \preturn{}(f) \pend \pif; # fin de l'algorithme
\quad # construction des deux sous-vecteurs de taille n1
\quad s := [ seq(f[2*k+1], k=0..n1-1) ];
\quad t := [ seq(f[2*k], k=1..n1) ];
\quad # calcul des deux sous-FFT :
\quad s := FFT_rec(s, signe, zeta^2 mod m);
\quad t := FFT_rec(t, signe, zeta^2 mod m);
\quad # mixage des deux résultats
\quad a := seq( s[k]+zeta^(-signe*(k-1))*t[k] mod m, k=1..n1 );
\quad b := seq( s[k]-zeta^(-signe*(k-1))*t[k] mod m, k=1..n1 );
\quad r := [a,b];
\quad \preturn{}(r);
\pend \pproc{}:}{%
}
\end{mapleinput}

\end{maplegroup}
\begin{maplegroup}
\begin{flushleft}
Procédure principale  (attention, le nom FFT est protégé en \Maple{} ...)
:
\end{flushleft}

\end{maplegroup}
\begin{maplegroup}
\begin{mapleinput}
\mapleinline{active}{1d}{xFFT := \pproc{}(f, signe)
\quad \plocal r;
\quad r := FFT_rec(f,signe,2);
\quad \pif signe=-1 \pthen r := 1/n*r mod m;
\quad \pelse r; \pend \pif
\pend \pproc{}: }{%
}
\end{mapleinput}

\end{maplegroup}
\begin{maplegroup}
\begin{flushleft}
Un test :
\end{flushleft}

\end{maplegroup}
\begin{maplegroup}
\begin{mapleinput}
\mapleinline{active}{1d}{hasard := rand(0..m-1):
x := [seq( hasard(), i=1..n )];
y := xFFT(x,+1); 
evalb( x = xFFT(y,-1) ); # On retombe bien sur nos pattes.}{%
}
\end{mapleinput}

\mapleresult
\begin{maplelatex}
\mapleinline{inert}{2d}{x := [179, 220, 230, 218, 49, 253, 197, 218, 67, 177, 136, 127, 190,
106, 210, 255];}{%
\[
x := [179, \,220, \,230, \,218, \,49, \,253, \,197, \,218, \,67, 
\,177, \,136, \,127, \,190, \,106, \,210, \,255]
\]
%
}
\end{maplelatex}

\begin{maplelatex}
\mapleinline{inert}{2d}{y := [5, 250, 28, 179, 190, 157, 195, 216, 198, 11, 13, 43, 5, 59,
49, 238];}{%
\[
y := [5, \,250, \,28, \,179, \,190, \,157, \,195, \,216, \,198, 
\,11, \,13, \,43, \,5, \,59, \,49, \,238]
\]
%
}
\end{maplelatex}

\begin{maplelatex}
\mapleinline{inert}{2d}{true;}{%
\[
\mathit{true}
\]
%
}
\end{maplelatex}

\end{maplegroup}

% \vskip -3mm
\end{footnotesize}
 
\caption{File \texttt{fft-ring.msw}}
 
\label{listing-fft-ring}
\end{listing}
 
% ------------------------------------------------- -----
% ------------------------------------------------- -----
% ------------------------------------------------- -----
% section - Multiplication of large integers                           
% ------------------------------------------------- -----
% ------------------------------------------------- -----
% ------------------------------------------------- -----
\section{Multiplication of large integers}
% \addcontentsline{toc}{section}{Multiplication of large integers}
\label{sect1-listing-mult-large-integers-ring}
 
 
\index{Integer!multiplication} The \Maple{} \texttt{mult-large-integers.mws} program allows you to calculate the product of two integers represented by their decomposition in a given base $ b $. This program uses the constants $ n $ and $ m $ as well as the function \texttt{xFFT} which is in the file \texttt{fft-ring.msw}, \ref{listing-fft-ring}. Here are the different things that can be found in this program. \begin{enumerate}
\item We first calculate an optimal value of $ b $, so as to satisfy $ n (b-1)^2 <m $.
\item Then several very useful functions are defined (to pass from the representation in the form of number to that in the form of vector).
\item The \texttt{prod \_entiers} function calculates the convolution product of the two vectors, then propagates the carry.
\item Finally, a test is performed. Of course, the usefulness of these functions is to multiply whole numbers that \Maple{} cannot handle (because they are too large), which is not the case in this test (because we make \Maple{check } that the product is right).
\end{enumerate}
 
 
 
\begin{listing} \begin{footnotesize}
\begin{maplegroup}
\begin{flushleft}
$b$ désigne la base de calcul. Il faut que 
\mapleinline{inert}{2d}{n(b-1)^2 < m;}{%
$n(b - 1)^{2} < m$%
}. 
\end{flushleft}

\end{maplegroup}
\begin{maplegroup}
\begin{mapleinput}
\mapleinline{active}{1d}{b := floor( evalf(sqrt(m/n))+1 ):}{%
}
\end{mapleinput}

\end{maplegroup}
\begin{maplegroup}
\begin{flushleft}
Calcule le produit point à point :
\end{flushleft}

\end{maplegroup}
\begin{maplegroup}
\begin{mapleinput}
\mapleinline{active}{1d}{cw_mult := \pproc{}(a,b)
\quad [seq( a[i]*b[i], i=1..n )]:
\pend \pproc{}:}{%
}
\end{mapleinput}

\end{maplegroup}
\begin{maplegroup}
\begin{flushleft}
Transforme un entier en vecteur :
\end{flushleft}

\end{maplegroup}
\begin{maplegroup}
\begin{mapleinput}
\mapleinline{active}{1d}{number2vector := \pproc{}(x)
\quad \plocal N, res, i, r, q, xx:
\quad N := floor( log(x)/log(b) )+1;
\quad res := []: xx := x:
\quad \pfor i \pfrom 1 \pto N \pdo
\quad \quad xx := iquo(xx,b,'r'):
\quad \quad res := [op(res), r]:
\quad \pend{}:
\quad \preturn{}(res):
\pend \pproc{}:}{%
}
\end{mapleinput}

\end{maplegroup}
\begin{maplegroup}
\begin{flushleft}
Transforme un vecteur en entier :
\end{flushleft}

\end{maplegroup}
\begin{maplegroup}
\begin{mapleinput}
\mapleinline{active}{1d}{vector2number := \pproc{}(v)
\quad add(v[k]*b^(k-1), k=1..nops(v));
\pend \pproc{}:}{%
}
\end{mapleinput}

\end{maplegroup}
\begin{maplegroup}
\begin{flushleft}
Calcule le produit de convolution :
\end{flushleft}

\end{maplegroup}
\begin{maplegroup}
\begin{mapleinput}
\mapleinline{active}{1d}{convol := \pproc{}(f,g)
\quad xFFT( cw_mult(xFFT(f,1),xFFT(g,1)), -1):
\pend \pproc{}:}{%
}
\end{mapleinput}

\end{maplegroup}
\begin{maplegroup}
\begin{flushleft}
Calcule le produit de deux entiers représentés sous forme de vecteurs
de taille 
\mapleinline{inert}{2d}{n;}{%
$n$%
}.
Attention, les 
\mapleinline{inert}{2d}{n;}{%
$n$%
}/2 dernières entrées des vecteurs doivent être nulles.
\end{flushleft}

\end{maplegroup}
\begin{maplegroup}
\begin{mapleinput}
\mapleinline{active}{1d}{prod_entiers := \pproc{}(x,y)
\quad \plocal res, i:
\quad res := convol(x,y):
\quad \pfor i \pfrom 1 \pto n-1 \pdo
\quad \quad res[i] := irem(res[i],b,'q'):
\quad \quad res[i+1] := res[i+1]+q;
\quad \pend{}:
\quad \preturn{}(res):
\pend \pproc{}:}{%
}
\end{mapleinput}

\end{maplegroup}
\begin{maplegroup}
\begin{flushleft}
Un test :
\end{flushleft}

\end{maplegroup}
\begin{maplegroup}
\begin{mapleinput}
\mapleinline{active}{1d}{hasard := rand(0..b-1):
xx := [seq( hasard(), i=1..n/2 ), seq(0, i=1..n/2)]; 
yy := [seq( hasard(), i=1..n/2 ), seq(0, i=1..n/2)]; 
x := vector2number(xx): y := vector2number(yy);
zz := prod_entiers(xx,yy);
evalb( vector2number(zz) = x*y ); # il doit y avoir égalité ...}{%
}
\end{mapleinput}

\mapleresult
\begin{maplelatex}
\mapleinline{inert}{2d}{xx := [4, 0, 0, 3, 3, 1, 0, 4, 0, 0, 0, 0, 0, 0, 0, 0];}{%
\[
\mathit{xx} := [4, \,0, \,0, \,3, \,3, \,1, \,0, \,4, \,0, \,0, 
\,0, \,0, \,0, \,0, \,0, \,0]
\]
%
}
\end{maplelatex}

\begin{maplelatex}
\mapleinline{inert}{2d}{yy := [3, 0, 4, 1, 4, 2, 3, 0, 0, 0, 0, 0, 0, 0, 0, 0];}{%
\[
\mathit{yy} := [3, \,0, \,4, \,1, \,4, \,2, \,3, \,0, \,0, \,0, 
\,0, \,0, \,0, \,0, \,0, \,0]
\]
%
}
\end{maplelatex}

\begin{maplelatex}
\mapleinline{inert}{2d}{y := 55853;}{%
\[
y := 55853
\]
%
}
\end{maplelatex}

\begin{maplelatex}
\mapleinline{inert}{2d}{zz := [2, 2, 1, 1, 3, 3, 2, 2, 1, 0, 3, 3, 2, 4, 2, 0];}{%
\[
\mathit{zz} := [2, \,2, \,1, \,1, \,3, \,3, \,2, \,2, \,1, \,0, 
\,3, \,3, \,2, \,4, \,2, \,0]
\]
%
}
\end{maplelatex}

\begin{maplelatex}
\mapleinline{inert}{2d}{true;}{%
\[
\mathit{true}
\]
%
}
\end{maplelatex}
\end{maplegroup}
% \vskip -3mm
\end{footnotesize}
 
\caption{File \texttt{mult-large-integers.msw}}
 
\label{listing-mult-large-integers}
\end{listing}
 
% ------------------------------------------------- -----
% ------------------------------------------------- -----
% ------------------------------------------------- -----
% section - Decoding BCH codes                           
% ------------------------------------------------- -----
% ------------------------------------------------- -----
% ------------------------------------------------- -----
\section{Decoding BCH codes}
% \addcontentsline{toc}{section}{Decoding BCH codes}
\label{sect1-listing-decoding-bch}
 
 
\index{BCH} \index{Decoding} This \Maple{} program uses the \texttt{FFT} function defined in the \ref{sect1-listing-transform-finite-body} program. This procedure should therefore be copied at the start of the program. The program has been split into three parts: \begin{rs}
\item Part 1 (program \ref{listing-decodage-bch-1}): search for the irreducible factors of $ X^n-1 $ on $ \FF_2 $, and construction of the generator polynomial of the BCH code.
\item Part 2 (program \ref{listing-decodage-bch-2}): definition of routines to manipulate code words both in the form of vectors and polynomials, to generate words at random.
\item Part 3 (program \ref{listing-decodage-bch-3}): the first part of the decoding algorithm, we calculate the values of $ \sigma_1, \ldots, \, \sigma_t $.
\item Part 4 (program \ref{listing-decodage-bch-4}): the second part of the decoding algorithm, we calculate the values of $ \wh{\epsilon_0}, \, \wh{\epsilon_{2t + 1}}, \ldots, \, \wh{\epsilon_{n-1}} $.
\end{rs}
 
 
 
\begin{listing} \begin{footnotesize}
\begin{maplegroup}
\begin{flushleft}
$r$ : degrés des facteurs irréductibles de 
\mapleinline{inert}{2d}{X^n-1;}{%
$X^{n} - 1$%
} sur 
\mapleinline{inert}{2d}{F[2];}{%
$\FF_2$%
} ; 
\mapleinline{inert}{2d}{t;}{%
$t$%
} : capacité de correction.
\end{flushleft}

\end{maplegroup}
\begin{maplegroup}
\begin{mapleinput}
\mapleinline{active}{1d}{\pwith{}(numtheory): \pwith{}(linalg):
n := 15: t := 3: delta:=2*t+1:}{%
}
\end{mapleinput}

\end{maplegroup}
\begin{maplegroup}
\begin{flushleft}
Liste des facteurs de 
\mapleinline{inert}{2d}{X^n-1;}{%
$X^{n} - 1$%
}choix d'un facteur irréductible de degré 
\mapleinline{inert}{2d}{r;}{%
$r$%
} et de la racine primive associée :
on constate que 
\mapleinline{inert}{2d}{r;}{%
$r$%
} est bien l'ordre de 
\mapleinline{inert}{2d}{p;}{%
$p$%
} dans $(\ZZ/n\ZZ)^*$.
\end{flushleft}

\end{maplegroup}
\begin{maplegroup}
\begin{mapleinput}
\mapleinline{active}{1d}{liste_div := op(Factor( X^n-1 ) mod 2 ):
P := liste_div[2];
alias( alpha = RootOf(P) ):}{%
}
\end{mapleinput}

\mapleresult
\begin{maplelatex}
\mapleinline{inert}{2d}{P := X^4+X^3+1;}{%
\[
P := X^{4} + X^{3} + 1
\]
%
}
\end{maplelatex}

\end{maplegroup}
\begin{maplegroup}
\begin{flushleft}
Calcule le polynôme générateur du code de distance prescrite 
\mapleinline{inert}{2d}{2*t+1;}{%
$2\,t + 1$%
}, le PPCM des polynômes minimaux des 
\mapleinline{inert}{2d}{alpha^i;}{%
$\alpha ^{i}$%
}, pour 
\mapleinline{inert}{2d}{i = 1;}{%
$i=1,\ldots,\,2t$%
} 
\end{flushleft}

\end{maplegroup}
\begin{maplegroup}
\begin{mapleinput}
\mapleinline{active}{1d}{calc_gen := \pproc{}()
\quad \plocal result, Q, i, liste_pol_rest: 
\quad result := P: # on sait déjà que P est dans le PPCM
\quad liste_pol_rest := \{liste_div\} minus \{P\}:
\quad # alpha^2 est racine de P, donc on peut le sauter
\quad \pfor i \pfrom 3 to 2*t \pdo  
\quad \pfor Q \pin liste_pol_rest \pdo
\quad \quad \pif Eval(Q, X=alpha^i) mod 2 = 0 \pthen
\quad \quad result := result*Q:
\quad \quad liste_pol_rest:=liste_pol_rest minus \{Q\}: break:
\quad \pend \pif: \pend \pdo: \pend \pdo:
\quad result := Expand(result) mod 2
\pend \pproc{}:}{%
}
\end{mapleinput}

\end{maplegroup}
\begin{maplegroup}
\begin{flushleft}
Polynôme générateur et dimension du code :
\end{flushleft}

\end{maplegroup}
\begin{maplegroup}
\begin{mapleinput}
\mapleinline{active}{1d}{G := sort( calc_gen() ); d := n - degree(G);}{%
}
\end{mapleinput}

\mapleresult
\begin{maplelatex}
\mapleinline{inert}{2d}{G := X^10+X^9+X^8+X^6+X^5+X^2+1;}{%
\[
G := X^{10} + X^{9} + X^{8} + X^{6} + X^{5} + X^{2} + 1
\]
%
}
\end{maplelatex}

\begin{maplelatex}
\mapleinline{inert}{2d}{d := 5;}{%
\[
d := 5
\]
%
}
\end{maplelatex}

\end{maplegroup}
% \vskip -3mm
\end{footnotesize}
 
\caption{File \texttt{decoding-bch.msw} part 1}
 
\label{listing-decodage-bch-1}
\end{listing}
 
\begin{listing} \begin{footnotesize}
\begin{maplegroup}
\begin{center}
\textbf{[...] Suite du script précédent}
\end{center}

\end{maplegroup}
\begin{maplegroup}
\begin{flushleft}
Calcule le mot de taille 
\mapleinline{inert}{2d}{n;}{%
$n$%
} (liste de 0/1) correspondant à un polynôme de degré 
\mapleinline{inert}{2d}{n-1;}{%
$n - 1$%
}
\end{flushleft}

\end{maplegroup}
\begin{maplegroup}
\begin{mapleinput}
\mapleinline{active}{1d}{Mot := \pproc{}(Q)
\quad [seq( coeff(Q, X,it), it=0..n-1 )]
\pend \pproc{}:}{%
}
\end{mapleinput}

\end{maplegroup}
\begin{maplegroup}
\begin{flushleft}
Calcule le polynôme de degré 
\mapleinline{inert}{2d}{n-1;}{%
$n - 1$%
} correspondant à un mot de taille 
\mapleinline{inert}{2d}{n;}{%
$n$%
}
\end{flushleft}

\end{maplegroup}
\begin{maplegroup}
\begin{mapleinput}
\mapleinline{active}{1d}{Pol := \pproc{}(mot)
\quad sum(mot[it]*X^(it-1), it=1..n);
\pend \pproc{}:}{%
}
\end{mapleinput}

\end{maplegroup}
\begin{maplegroup}
\begin{flushleft}
Calcule le syndrôme d'indice 
\mapleinline{inert}{2d}{i;}{%
$i$%
}, i.e. 
\mapleinline{inert}{2d}{P(alpha^i);}{%
$\mathrm{P}(\alpha ^{i})$%
} :
\end{flushleft}

\end{maplegroup}
\begin{maplegroup}
\begin{mapleinput}
\mapleinline{active}{1d}{Syndi := \pproc{}(pol, i)
\quad Eval(pol, X = alpha^i) mod 2;
\pend \pproc{}: }{%
}
\end{mapleinput}

\end{maplegroup}
\begin{maplegroup}
\begin{flushleft}
Calcule un vecteur aléatoire avec \texttt{nb\_erreurs} erreurs
\end{flushleft}

\end{maplegroup}
\begin{maplegroup}
\begin{mapleinput}
\mapleinline{active}{1d}{Aleat := \pproc{}(nb_erreurs)
\quad \plocal hasard:
\quad hasard := rand(1..(n-1)):
\quad Mot( add(X^hasard(), i=1..nb_erreurs) mod 2 );    
\pend \pproc{}:}{%
}
\end{mapleinput}

\end{maplegroup}
\begin{maplegroup}
\begin{flushleft}
Calcule un mot du code au hasard
\end{flushleft}

\end{maplegroup}
\begin{maplegroup}
\begin{mapleinput}
\mapleinline{active}{1d}{MotCode := \pproc{}{}()
\quad \plocal Q;
\quad Q := Randpoly(d-1, X) mod 2;
\quad Q := Expand( Q*G ) mod 2;
\quad Mot(Q);
\pend \pproc{}:}{%
}
\end{mapleinput}

\end{maplegroup}
\begin{maplegroup}
\begin{flushleft}
On simule une transmission avec erreur :
\end{flushleft}

\end{maplegroup}
\begin{maplegroup}
\begin{mapleinput}
\mapleinline{active}{1d}{mot_code := MotCode();
mot_transmis := mot_code + Aleat(3) mod 2;
p_recu := Pol(mot_transmis);}{%
}
\end{mapleinput}

\mapleresult
\begin{maplelatex}
\mapleinline{inert}{2d}{mot_code := [0, 1, 1, 1, 1, 0, 0, 0, 1, 0, 0, 1, 1, 0, 1];}{%
\[
\mathit{mot\_code} := [0, \,1, \,1, \,1, \,1, \,0, \,0, \,0, \,1
, \,0, \,0, \,1, \,1, \,0, \,1]
\]
%
}
\end{maplelatex}

\begin{maplelatex}
\mapleinline{inert}{2d}{mot_transmis := [0, 1, 1, 1, 1, 0, 0, 0, 1, 0, 0, 0, 0, 0, 0];}{%
\[
\mathit{mot\_transmis} := [0, \,1, \,1, \,1, \,1, \,0, \,0, \,0, 
\,1, \,0, \,0, \,0, \,0, \,0, \,0]
\]
%
}
\end{maplelatex}

\begin{maplelatex}
\mapleinline{inert}{2d}{p_recu := X+X^2+X^3+X^4+X^8;}{%
\[
\mathit{p\_recu} := X + X^{2} + X^{3} + X^{4} + X^{8}
\]
%
}
\end{maplelatex}

\end{maplegroup}
% \vskip -3mm
\end{footnotesize}
 
\caption{File \texttt{decoding-bch.msw} part 2}
 
\label{listing-decodage-bch-2}
\end{listing}
 
\begin{listing} \begin{footnotesize}
\begin{maplegroup}
\begin{center}
\textbf{[...] Suite du script précédent}
\end{center}

\end{maplegroup}
\begin{maplegroup}
\begin{flushleft}
\textbf{{\large 1ère partie : }}Résolution des équations pour 
\mapleinline{inert}{2d}{i = n-t;}{%
$i=n - t$%
} ... 
\mapleinline{inert}{2d}{n-1;}{%
$n - 1$%
} pour trouver 
\mapleinline{inert}{2d}{sigma;}{%
$\sigma $%
}[1] ... 
\mapleinline{inert}{2d}{sigma;}{%
$\sigma $%
}[t]
\end{flushleft}

\end{maplegroup}
\begin{maplegroup}
\begin{flushleft}
Calcule de l'équation polynomiale à résoudre (attention, on note la 
\mapleinline{inert}{2d}{epsilon;}{%
$\varepsilon $%
} \textit{transformée de Fourier} de l'erreur) :
\end{flushleft}

\end{maplegroup}
\begin{maplegroup}
\begin{mapleinput}
\mapleinline{active}{1d}{eqn := (1+add(sigma[i]*Z^i,i=1..t))*
\quad \quad \quad (add(epsilon[n-i]*Z^i,i=1..n)):
eqn := rem(eqn,Z^n-1,Z,'q');  # l'équation est modulo Z^n-1}{%
}
\end{mapleinput}

\mapleresult
\begin{maplelatex}
\mapleinline{inert}{2d}{eqn :=
(sigma[1]*epsilon[2]+sigma[3]*epsilon[4]+epsilon[1]+sigma[2]*epsilon[3
])*Z^14+(sigma[3]*epsilon[5]+epsilon[2]+sigma[1]*epsilon[3]+sigma[2]*e
psilon[4])*Z^13+(epsilon[3]+sigma[2]*epsilon[5]+sigma[1]*epsilon[4]+si
gma[3]*epsilon[6])*Z^12+(sigma[3]*epsilon[7]+sigma[2]*epsilon[6]+sigma
[1]*epsilon[5]+epsilon[4])*Z^11+(sigma[1]*epsilon[6]+epsilon[5]+sigma[
3]*epsilon[8]+sigma[2]*epsilon[7])*Z^10+(sigma[1]*epsilon[7]+sigma[3]*
epsilon[9]+epsilon[6]+sigma[2]*epsilon[8])*Z^9+(sigma[3]*epsilon[10]+s
igma[2]*epsilon[9]+sigma[1]*epsilon[8]+epsilon[7])*Z^8+(sigma[1]*epsil
on[9]+epsilon[8]+sigma[2]*epsilon[10]+sigma[3]*epsilon[11])*Z^7+(epsil
on[9]+sigma[1]*epsilon[10]+sigma[2]*epsilon[11]+sigma[3]*epsilon[12])*
Z^6+(sigma[2]*epsilon[12]+epsilon[10]+sigma[1]*epsilon[11]+sigma[3]*ep
silon[13])*Z^5+(sigma[1]*epsilon[12]+epsilon[11]+sigma[3]*epsilon[14]+
sigma[2]*epsilon[13])*Z^4+(sigma[1]*epsilon[13]+sigma[2]*epsilon[14]+e
psilon[12]+sigma[3]*epsilon[0])*Z^3+(sigma[1]*epsilon[14]+epsilon[13]+
sigma[3]*epsilon[1]+sigma[2]*epsilon[0])*Z^2+(epsilon[14]+sigma[1]*eps
ilon[0]+sigma[3]*epsilon[2]+sigma[2]*epsilon[1])*Z+sigma[3]*epsilon[3]
+epsilon[0]+sigma[2]*epsilon[2]+sigma[1]*epsilon[1];}{%
\maplemultiline{
\mathit{eqn} := ({\sigma _{1}}\,{\varepsilon _{2}} + {\sigma _{3}
}\,{\varepsilon _{4}} + {\varepsilon _{1}} + {\sigma _{2}}\,{
\varepsilon _{3}})\,Z^{14} + ({\sigma _{3}}\,{\varepsilon _{5}}
 + {\varepsilon _{2}} + {\sigma _{1}}\,{\varepsilon _{3}} + {
\sigma _{2}}\,{\varepsilon _{4}})\,Z^{13} \\
\mbox{} + ({\varepsilon _{3}} + {\sigma _{2}}\,{\varepsilon _{5}}
 + {\sigma _{1}}\,{\varepsilon _{4}} + {\sigma _{3}}\,{
\varepsilon _{6}})\,Z^{12} + ({\sigma _{3}}\,{\varepsilon _{7}}
 + {\sigma _{2}}\,{\varepsilon _{6}} + {\sigma _{1}}\,{
\varepsilon _{5}} + {\varepsilon _{4}})\,Z^{11} \\
\mbox{} + ({\sigma _{1}}\,{\varepsilon _{6}} + {\varepsilon _{5}}
 + {\sigma _{3}}\,{\varepsilon _{8}} + {\sigma _{2}}\,{
\varepsilon _{7}})\,Z^{10} + ({\sigma _{1}}\,{\varepsilon _{7}}
 + {\sigma _{3}}\,{\varepsilon _{9}} + {\varepsilon _{6}} + {
\sigma _{2}}\,{\varepsilon _{8}})\,Z^{9} \\
\mbox{} + ({\sigma _{3}}\,{\varepsilon _{10}} + {\sigma _{2}}\,{
\varepsilon _{9}} + {\sigma _{1}}\,{\varepsilon _{8}} + {
\varepsilon _{7}})\,Z^{8} + ({\sigma _{1}}\,{\varepsilon _{9}} + 
{\varepsilon _{8}} + {\sigma _{2}}\,{\varepsilon _{10}} + {\sigma
 _{3}}\,{\varepsilon _{11}})\,Z^{7} \\
\mbox{} + ({\varepsilon _{9}} + {\sigma _{1}}\,{\varepsilon _{10}
} + {\sigma _{2}}\,{\varepsilon _{11}} + {\sigma _{3}}\,{
\varepsilon _{12}})\,Z^{6} + ({\sigma _{2}}\,{\varepsilon _{12}}
 + {\varepsilon _{10}} + {\sigma _{1}}\,{\varepsilon _{11}} + {
\sigma _{3}}\,{\varepsilon _{13}})\,Z^{5} \\
\mbox{} + ({\sigma _{1}}\,{\varepsilon _{12}} + {\varepsilon _{11
}} + {\sigma _{3}}\,{\varepsilon _{14}} + {\sigma _{2}}\,{
\varepsilon _{13}})\,Z^{4} + ({\sigma _{1}}\,{\varepsilon _{13}}
 + {\sigma _{2}}\,{\varepsilon _{14}} + {\varepsilon _{12}} + {
\sigma _{3}}\,{\varepsilon _{0}})\,Z^{3} \\
\mbox{} + ({\sigma _{1}}\,{\varepsilon _{14}} + {\varepsilon _{13
}} + {\sigma _{3}}\,{\varepsilon _{1}} + {\sigma _{2}}\,{
\varepsilon _{0}})\,Z^{2} + ({\varepsilon _{14}} + {\sigma _{1}}
\,{\varepsilon _{0}} + {\sigma _{3}}\,{\varepsilon _{2}} + {
\sigma _{2}}\,{\varepsilon _{1}})\,Z + {\sigma _{3}}\,{
\varepsilon _{3}} + {\varepsilon _{0}} \\
\mbox{} + {\sigma _{2}}\,{\varepsilon _{2}} + {\sigma _{1}}\,{
\varepsilon _{1}} }
%
}
\end{maplelatex}

\end{maplegroup}
\begin{maplegroup}
\begin{flushleft}
Calcule les équations à résoudre, liste les valeurs de 
\mapleinline{inert}{2d}{epsilon;}{%
$\varepsilon $%
} connues, pour 
\mapleinline{inert}{2d}{i = 1;}{%
$i=1$%
} ... 
\mapleinline{inert}{2d}{2*t;}{%
$2\,t$%
}, puis évalue les équations :
\end{flushleft}

\end{maplegroup}
\begin{maplegroup}
\begin{mapleinput}
\mapleinline{active}{1d}{list_eqn1 := \{seq( coeff(eqn,Z,i), i=n-t..n-1 )\}:
epsilon_connu := \{seq( epsilon[i] = Syndi(p_recu,i), i=1..2*t )\};
eqn_eval1 := eval(list_eqn1, epsilon_connu);}{%
}
\end{mapleinput}

\mapleresult
\begin{maplelatex}
\mapleinline{inert}{2d}{epsilon_connu := \{epsilon[2] = alpha^3+alpha^2, epsilon[1] =
alpha^3, epsilon[6] = alpha^3+1, epsilon[5] = 1, epsilon[4] =
alpha^3+alpha^2+alpha+1, epsilon[3] = alpha^3+alpha+1\};}{%
\maplemultiline{
\mathit{epsilon\_connu} :=  \\
\{{\varepsilon _{2}}=\alpha ^{3} + \alpha ^{2}, \,{\varepsilon _{
1}}=\alpha ^{3}, \,{\varepsilon _{6}}=\alpha ^{3} + 1, \,{
\varepsilon _{5}}=1, \,{\varepsilon _{4}}=\alpha ^{3} + \alpha ^{
2} + \alpha  + 1, \,{\varepsilon _{3}}=\alpha ^{3} + \alpha  + 1
\} }
%
}
\end{maplelatex}

\begin{maplelatex}
\mapleinline{inert}{2d}{eqn_eval1 :=
\{sigma[1]*(alpha^3+alpha^2+alpha+1)+alpha^3+alpha+1+sigma[3]*(alpha^3
+1)+sigma[2],
sigma[2]*(alpha^3+alpha^2+alpha+1)+alpha^3+alpha^2+sigma[3]+sigma[1]*(
alpha^3+alpha+1),
alpha^3+sigma[3]*(alpha^3+alpha^2+alpha+1)+sigma[2]*(alpha^3+alpha+1)+
sigma[1]*(alpha^3+alpha^2)\};}{%
\maplemultiline{
\mathit{eqn\_eval1} := \{{\sigma _{1}}\,(\alpha ^{3} + \alpha ^{2
} + \alpha  + 1) + \alpha ^{3} + \alpha  + 1 + {\sigma _{3}}\,(
\alpha ^{3} + 1) + {\sigma _{2}},  \\
{\sigma _{2}}\,(\alpha ^{3} + \alpha ^{2} + \alpha  + 1) + \alpha
 ^{3} + \alpha ^{2} + {\sigma _{3}} + {\sigma _{1}}\,(\alpha ^{3}
 + \alpha  + 1),  \\
\alpha ^{3} + {\sigma _{3}}\,(\alpha ^{3} + \alpha ^{2} + \alpha 
 + 1) + {\sigma _{2}}\,(\alpha ^{3} + \alpha  + 1) + {\sigma _{1}
}\,(\alpha ^{3} + \alpha ^{2})\} }
%
}
\end{maplelatex}

\end{maplegroup}
\begin{maplegroup}
\begin{flushleft}
Met sous forme matricielle les équations :
\end{flushleft}

\end{maplegroup}
\begin{maplegroup}
\begin{mapleinput}
\mapleinline{active}{1d}{m1 := matrix(t,t):
b1 := vector(t):
i := 1:
\pfor eq \pin eqn_eval1 \pdo
\quad \pfor j \pfrom 1 \pto t \pdo
\quad \quad m1[i,j] := coeff(eq,sigma[j],1);
\quad \pend \pdo:
\quad b1[i] := eval( eq, [seq(sigma[k]=0,k=1..t)] );
\quad i := i+1:
\pend \pdo:}{%
}
\end{mapleinput}

\end{maplegroup}
\begin{maplegroup}
\begin{flushleft}
Calcule les valeurs de 
\mapleinline{inert}{2d}{sigma;}{%
$\sigma $%
} en résolvant le système :
\end{flushleft}

\end{maplegroup}
\begin{maplegroup}
\begin{mapleinput}
\mapleinline{active}{1d}{sigma_val := Linsolve(m1,b1) mod 2:
sigma_connu := \{ seq(sigma[i]=sigma_val[i], i = 1..t) \};}{%
}
\end{mapleinput}

\mapleresult
\begin{maplelatex}
\mapleinline{inert}{2d}{sigma_connu := \{sigma[1] = alpha^3+1, sigma[2] =
alpha^3+alpha^2+alpha, sigma[3] = alpha^2+1\};}{%
\[
\mathit{sigma\_connu} := \{{\sigma _{1}}=\alpha ^{3} + 1, \,{
\sigma _{2}}=\alpha ^{3} + \alpha ^{2} + \alpha , \,{\sigma _{3}}
=\alpha ^{2} + 1\}
\]
%
}
\end{maplelatex}

\end{maplegroup}
% \vskip -3mm
\end{footnotesize}
 
\caption{File \texttt{decoding-bch.msw} part 3}
 
\label{listing-decodage-bch-3}
\end{listing}
 
\begin{listing} \begin{footnotesize}
\begin{maplegroup}
\begin{flushleft}
\textbf{{\large 2e partie : }}Résolution des équations pour 
\mapleinline{inert}{2d}{i = 0;}{%
$i=0$%
} ... 
\mapleinline{inert}{2d}{n-2*t-1;}{%
$n - 2\,t - 1$%
} pour trouver 
\mapleinline{inert}{2d}{epsilon;}{%
$\varepsilon $%
}[0], 
\mapleinline{inert}{2d}{epsilon;}{%
$\varepsilon $%
}[2t+1] ... 
\mapleinline{inert}{2d}{epsilon;}{%
$\varepsilon $%
}[n-1]
\end{flushleft}

\end{maplegroup}
\begin{maplegroup}
\begin{flushleft}
Calcule les équations pour 
\mapleinline{inert}{2d}{i = 0;}{%
$i=0$%
} ... 
\mapleinline{inert}{2d}{n-2*t-1;}{%
$n - 2\,t - 1$%
}, puis les évalue :
\end{flushleft}

\end{maplegroup}
\begin{maplegroup}
\begin{mapleinput}
\mapleinline{active}{1d}{list_eqn2 := \{seq( coeff(eqn,Z,i), i=0..n-2*t-1 )\}:
eqn_eval2 := eval(list_eqn2, epsilon_connu):
eqn_eval2 := eval(eqn_eval2, sigma_connu);}{%
}
\end{mapleinput}

\mapleresult
\begin{maplelatex}
\mapleinline{inert}{2d}{eqn_eval2 :=
\{(alpha^3+alpha^2+alpha)*epsilon[14]+epsilon[12]+(alpha^3+1)*epsilon[
13]+(alpha^2+1)*epsilon[0],
epsilon[14]+(alpha^3+1)*epsilon[0]+(alpha^2+1)*(alpha^3+alpha^2)+(alph
a^3+alpha^2+alpha)*alpha^3,
epsilon[13]+(alpha^3+1)*epsilon[14]+(alpha^3+alpha^2+alpha)*epsilon[0]
+(alpha^2+1)*alpha^3,
epsilon[0]+(alpha^3+alpha^2+alpha)*(alpha^3+alpha^2)+(alpha^3+alpha+1)
*(alpha^2+1)+(alpha^3+1)*alpha^3,
epsilon[11]+(alpha^2+1)*epsilon[14]+(alpha^3+alpha^2+alpha)*epsilon[13
]+(alpha^3+1)*epsilon[12],
epsilon[10]+(alpha^3+1)*epsilon[11]+(alpha^2+1)*epsilon[13]+(alpha^3+a
lpha^2+alpha)*epsilon[12],
epsilon[8]+(alpha^3+1)*epsilon[9]+(alpha^3+alpha^2+alpha)*epsilon[10]+
(alpha^2+1)*epsilon[11],
(alpha^2+1)*epsilon[12]+(alpha^3+alpha^2+alpha)*epsilon[11]+(alpha^3+1
)*epsilon[10]+epsilon[9],
(alpha^2+1)*epsilon[10]+epsilon[7]+(alpha^3+alpha^2+alpha)*epsilon[9]+
(alpha^3+1)*epsilon[8]\};}{%
\maplemultiline{
\mathit{eqn\_eval2} := \{\mathrm{\%1}\,{\varepsilon _{14}} + {
\varepsilon _{12}} + (\alpha ^{3} + 1)\,{\varepsilon _{13}} + (
\alpha ^{2} + 1)\,{\varepsilon _{0}},  \\
{\varepsilon _{14}} + (\alpha ^{3} + 1)\,{\varepsilon _{0}} + (
\alpha ^{2} + 1)\,(\alpha ^{3} + \alpha ^{2}) + \mathrm{\%1}\,
\alpha ^{3},  \\
{\varepsilon _{13}} + (\alpha ^{3} + 1)\,{\varepsilon _{14}} + 
\mathrm{\%1}\,{\varepsilon _{0}} + (\alpha ^{2} + 1)\,\alpha ^{3}
,  \\
{\varepsilon _{0}} + \mathrm{\%1}\,(\alpha ^{3} + \alpha ^{2}) + 
(\alpha ^{3} + \alpha  + 1)\,(\alpha ^{2} + 1) + (\alpha ^{3} + 1
)\,\alpha ^{3},  \\
{\varepsilon _{11}} + (\alpha ^{2} + 1)\,{\varepsilon _{14}} + 
\mathrm{\%1}\,{\varepsilon _{13}} + (\alpha ^{3} + 1)\,{
\varepsilon _{12}}, \,{\varepsilon _{10}} + (\alpha ^{3} + 1)\,{
\varepsilon _{11}} + (\alpha ^{2} + 1)\,{\varepsilon _{13}} + 
\mathrm{\%1}\,{\varepsilon _{12}},  \\
{\varepsilon _{8}} + (\alpha ^{3} + 1)\,{\varepsilon _{9}} + 
\mathrm{\%1}\,{\varepsilon _{10}} + (\alpha ^{2} + 1)\,{
\varepsilon _{11}}, \,(\alpha ^{2} + 1)\,{\varepsilon _{12}} + 
\mathrm{\%1}\,{\varepsilon _{11}} + (\alpha ^{3} + 1)\,{
\varepsilon _{10}} + {\varepsilon _{9}},  \\
(\alpha ^{2} + 1)\,{\varepsilon _{10}} + {\varepsilon _{7}} + 
\mathrm{\%1}\,{\varepsilon _{9}} + (\alpha ^{3} + 1)\,{
\varepsilon _{8}}\} \\
\mathrm{\%1} := \alpha ^{3} + \alpha ^{2} + \alpha  }
%
}
\end{maplelatex}

\end{maplegroup}
\begin{maplegroup}
\begin{flushleft}
Met sous forme matricielle les équations :
\end{flushleft}

\end{maplegroup}
\begin{maplegroup}
\begin{mapleinput}
\mapleinline{active}{1d}{# les indices de epsilon a calculer
epsilon_indices := [0,seq(i, i=2*t+1..n-1)]: 
m2 := matrix(n-2*t,n-2*t):
b2 := vector(n-2*t):
i := 1:
\pfor eq \pin eqn_eval2 \pdo
\quad j:= 1:
\quad \pfor index \pin epsilon_indices \pdo
\quad \quad m2[i,j] := coeff(eq,epsilon[index],1):
\quad \quad j := j+1;
\quad \pend \pdo:
\quad b2[i]:=eval(eq,[epsilon[0]=0,seq(epsilon[k]=0,k=2*t+1..n-1)]);
\quad i := i+1:
\pend \pdo:}{%
}
\end{mapleinput}

\end{maplegroup}
\begin{maplegroup}
\begin{flushleft}
Calcule les valeurs de 
\mapleinline{inert}{2d}{epsilon;}{%
$\varepsilon $%
}[0], 
\mapleinline{inert}{2d}{epsilon;}{%
$\varepsilon $%
}[2t+1] ... 
\mapleinline{inert}{2d}{epsilon;}{%
$\varepsilon $%
}[n-1], puis regroupe toutes les valeurs : 
\end{flushleft}

\end{maplegroup}
\begin{maplegroup}
\begin{mapleinput}
\mapleinline{active}{1d}{epsilon_val := Linsolve(m2,b2) mod 2:
epsilon_val := [epsilon_val[1], seq(Syndi(p_recu,it),it=1..2*t),
seq(epsilon_val[it],it=2..n-2*t)];}{%
}
\end{mapleinput}

\mapleresult
\begin{maplelatex}
\mapleinline{inert}{2d}{epsilon_val := [1, alpha^3, alpha^3+alpha^2, alpha^3+alpha+1,
alpha^3+alpha^2+alpha+1, 1, alpha^3+1, alpha^3+alpha+1, alpha^3+alpha,
alpha^3+alpha^2+alpha, 1, alpha^3+alpha^2+alpha, alpha^3+alpha^2+1,
alpha^3+alpha^2+1, alpha^3+1];}{%
\maplemultiline{
\mathit{epsilon\_val} := [1, \,\alpha ^{3}, \,\alpha ^{3} + 
\alpha ^{2}, \,\alpha ^{3} + \alpha  + 1, \,\alpha ^{3} + \alpha 
^{2} + \alpha  + 1, \,1, \,\alpha ^{3} + 1, \,\alpha ^{3} + 
\alpha  + 1, \,\alpha ^{3} + \alpha ,  \\
\alpha ^{3} + \alpha ^{2} + \alpha , \,1, \,\alpha ^{3} + \alpha 
^{2} + \alpha , \,\alpha ^{3} + \alpha ^{2} + 1, \,\alpha ^{3} + 
\alpha ^{2} + 1, \,\alpha ^{3} + 1] }
%
}
\end{maplelatex}

\end{maplegroup}
\begin{maplegroup}
\begin{flushleft}
On peut maintenant déterminer l'erreur par transformée de Fourier
inverse :
\end{flushleft}

\end{maplegroup}
\begin{maplegroup}
\begin{mapleinput}
\mapleinline{active}{1d}{erreurs := Normal( TFD(epsilon_val,-1) ) mod 2;
mot_corrige := mot_transmis - erreurs mod 2:
evalb( mot_corrige = mot_code );}{%
}
\end{mapleinput}

\mapleresult
\begin{maplelatex}
\mapleinline{inert}{2d}{erreurs := [0, 0, 0, 0, 0, 0, 0, 0, 0, 0, 0, 1, 1, 0, 1];}{%
\[
\mathit{erreurs} := [0, \,0, \,0, \,0, \,0, \,0, \,0, \,0, \,0, 
\,0, \,0, \,1, \,1, \,0, \,1]
\]
%
}
\end{maplelatex}

\begin{maplelatex}
\mapleinline{inert}{2d}{true;}{%
\[
\mathit{true}
\]
%
}
\end{maplelatex}

\end{maplegroup}
% \vskip -3mm
\end{footnotesize}
 
\caption{File \texttt{decoding-bch.msw} part 4}
 
\label{listing-decodage-bch-4}
\end{listing}

