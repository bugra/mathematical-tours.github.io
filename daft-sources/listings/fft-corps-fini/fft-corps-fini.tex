\begin{maplegroup}
\begin{flushleft}
Les paramètres pour faire un TFD de taille 
\mapleinline{inert}{2d}{n;}{%
$n$%
} fixé sur 
\mapleinline{inert}{2d}{Fp;}{%
$\FF_p$%
} :
\end{flushleft}

\end{maplegroup}
\begin{maplegroup}
\begin{mapleinput}
\mapleinline{active}{1d}{\pwith{}(numtheory): n := 16: p := 3:}{%
}
\end{mapleinput}

\end{maplegroup}
\begin{maplegroup}
\begin{flushleft}
Liste des facteurs de 
\mapleinline{inert}{2d}{X^n-1;}{%
$X^{n} - 1$%
}. Choix d'un facteur irréductible de degré 
\mapleinline{inert}{2d}{r;}{%
$r$%
} et de la racine primive associée :
on constate que 
\mapleinline{inert}{2d}{r;}{%
$r$%
} est bien l'ordre de 
\mapleinline{inert}{2d}{p;}{%
$p$%
} dans $(\ZZ/n\ZZ)^*$.
\end{flushleft}

\end{maplegroup}
\begin{maplegroup}
\begin{mapleinput}
\mapleinline{active}{1d}{liste_div := op(Factor( cyclotomic(n,X) ) mod p );
P := liste_div[1];
alias( alpha = RootOf(P) ):}{%
}
\end{mapleinput}

\mapleresult
\begin{maplelatex}
\mapleinline{inert}{2d}{liste_div := X^4+2*X^2+2, X^4+X^2+2;}{%
\[
\mathit{liste\_div} := X^{4} + 2\,X^{2} + 2, \,X^{4} + X^{2} + 2
\]
%
}
\end{maplelatex}

\begin{maplelatex}
\mapleinline{inert}{2d}{P := X^4+2*X^2+2;}{%
\[
P := X^{4} + 2\,X^{2} + 2
\]
%
}
\end{maplelatex}

\end{maplegroup}
\begin{maplegroup}
\begin{flushleft}
Transformée de Fourier, version 
\mapleinline{inert}{2d}{O(n^2);}{%
$\mathrm{O}(n^{2})$%
} :
\end{flushleft}

\end{maplegroup}
\begin{maplegroup}
\begin{mapleinput}
\mapleinline{active}{1d}{TFD := \pproc{}(f, signe)
\quad \plocal res, formule;
\quad # pour plus de lisibilité, on écrit à part la formule de TFD :
\quad formule := 'f[l+1]*alpha^(-signe*(k-1)*l)';
\quad res := [ seq( sum( formule, 'l'=0..n-1 ) mod p , k=1..n)  ];     
\quad \pif signe=-1 \pthen res := 1/n*res mod p \pend \pif;
\quad \preturn{}(Normal(res) mod p);
\pend \pproc{}:}{%
}
\end{mapleinput}

\end{maplegroup}
\begin{maplegroup}
\begin{flushleft}
Test simple :
\end{flushleft}

\end{maplegroup}
\begin{maplegroup}
\begin{mapleinput}
\mapleinline{active}{1d}{hasard := rand(0..(p-1)):
x := [seq( hasard(), i=1..n )];
y := TFD(x,1);  # TFD(x) n'est plus à coefficients dans F_2.
evalb( x = TFD(y,-1) ); # Mais on retombe bien sur nos pattes.}{%
}
\end{mapleinput}

\mapleresult
\begin{maplelatex}
\mapleinline{inert}{2d}{x := [0, 2, 0, 2, 1, 2, 2, 2, 1, 1, 1, 0, 0, 2, 2, 1];}{%
\[
x := [0, \,2, \,0, \,2, \,1, \,2, \,2, \,2, \,1, \,1, \,1, \,0, 
\,0, \,2, \,2, \,1]
\]
%
}
\end{maplelatex}

\begin{maplelatex}
\mapleinline{inert}{2d}{y := [1, 2*alpha^3+2*alpha+alpha^2+2, 1, 2*alpha^3+alpha+2*alpha^2,
alpha^2+1, alpha^3+alpha, 1, 2*alpha^3+alpha, 1,
alpha^3+alpha+alpha^2+2, 2, alpha^3+2*alpha+2*alpha^2, 2*alpha^2+2,
2*alpha^3+2*alpha, 2, alpha^3+2*alpha];}{%
\maplemultiline{
y := [1, \,2\,\alpha ^{3} + 2\,\alpha  + \alpha ^{2} + 2, \,1, \,
2\,\alpha ^{3} + \alpha  + 2\,\alpha ^{2}, \,\alpha ^{2} + 1, \,
\alpha ^{3} + \alpha , \,1, \,2\,\alpha ^{3} + \alpha , \,1,  \\
\alpha ^{3} + \alpha  + \alpha ^{2} + 2, \,2, \,\alpha ^{3} + 2\,
\alpha  + 2\,\alpha ^{2}, \,2\,\alpha ^{2} + 2, \,2\,\alpha ^{3}
 + 2\,\alpha , \,2, \,\alpha ^{3} + 2\,\alpha ] }
%
}
\end{maplelatex}

\begin{maplelatex}
\mapleinline{inert}{2d}{true;}{%
\[
\mathit{true}
\]
%
}
\end{maplelatex}

\end{maplegroup}
