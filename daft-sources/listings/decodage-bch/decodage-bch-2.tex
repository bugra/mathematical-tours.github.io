\begin{maplegroup}
\begin{center}
\textbf{[...] Suite du script précédent}
\end{center}

\end{maplegroup}
\begin{maplegroup}
\begin{flushleft}
Calcule le mot de taille 
\mapleinline{inert}{2d}{n;}{%
$n$%
} (liste de 0/1) correspondant à un polynôme de degré 
\mapleinline{inert}{2d}{n-1;}{%
$n - 1$%
}
\end{flushleft}

\end{maplegroup}
\begin{maplegroup}
\begin{mapleinput}
\mapleinline{active}{1d}{Mot := \pproc{}(Q)
\quad [seq( coeff(Q, X,it), it=0..n-1 )]
\pend \pproc{}:}{%
}
\end{mapleinput}

\end{maplegroup}
\begin{maplegroup}
\begin{flushleft}
Calcule le polynôme de degré 
\mapleinline{inert}{2d}{n-1;}{%
$n - 1$%
} correspondant à un mot de taille 
\mapleinline{inert}{2d}{n;}{%
$n$%
}
\end{flushleft}

\end{maplegroup}
\begin{maplegroup}
\begin{mapleinput}
\mapleinline{active}{1d}{Pol := \pproc{}(mot)
\quad sum(mot[it]*X^(it-1), it=1..n);
\pend \pproc{}:}{%
}
\end{mapleinput}

\end{maplegroup}
\begin{maplegroup}
\begin{flushleft}
Calcule le syndrôme d'indice 
\mapleinline{inert}{2d}{i;}{%
$i$%
}, i.e. 
\mapleinline{inert}{2d}{P(alpha^i);}{%
$\mathrm{P}(\alpha ^{i})$%
} :
\end{flushleft}

\end{maplegroup}
\begin{maplegroup}
\begin{mapleinput}
\mapleinline{active}{1d}{Syndi := \pproc{}(pol, i)
\quad Eval(pol, X = alpha^i) mod 2;
\pend \pproc{}: }{%
}
\end{mapleinput}

\end{maplegroup}
\begin{maplegroup}
\begin{flushleft}
Calcule un vecteur aléatoire avec \texttt{nb\_erreurs} erreurs
\end{flushleft}

\end{maplegroup}
\begin{maplegroup}
\begin{mapleinput}
\mapleinline{active}{1d}{Aleat := \pproc{}(nb_erreurs)
\quad \plocal hasard:
\quad hasard := rand(1..(n-1)):
\quad Mot( add(X^hasard(), i=1..nb_erreurs) mod 2 );    
\pend \pproc{}:}{%
}
\end{mapleinput}

\end{maplegroup}
\begin{maplegroup}
\begin{flushleft}
Calcule un mot du code au hasard
\end{flushleft}

\end{maplegroup}
\begin{maplegroup}
\begin{mapleinput}
\mapleinline{active}{1d}{MotCode := \pproc{}{}()
\quad \plocal Q;
\quad Q := Randpoly(d-1, X) mod 2;
\quad Q := Expand( Q*G ) mod 2;
\quad Mot(Q);
\pend \pproc{}:}{%
}
\end{mapleinput}

\end{maplegroup}
\begin{maplegroup}
\begin{flushleft}
On simule une transmission avec erreur :
\end{flushleft}

\end{maplegroup}
\begin{maplegroup}
\begin{mapleinput}
\mapleinline{active}{1d}{mot_code := MotCode();
mot_transmis := mot_code + Aleat(3) mod 2;
p_recu := Pol(mot_transmis);}{%
}
\end{mapleinput}

\mapleresult
\begin{maplelatex}
\mapleinline{inert}{2d}{mot_code := [0, 1, 1, 1, 1, 0, 0, 0, 1, 0, 0, 1, 1, 0, 1];}{%
\[
\mathit{mot\_code} := [0, \,1, \,1, \,1, \,1, \,0, \,0, \,0, \,1
, \,0, \,0, \,1, \,1, \,0, \,1]
\]
%
}
\end{maplelatex}

\begin{maplelatex}
\mapleinline{inert}{2d}{mot_transmis := [0, 1, 1, 1, 1, 0, 0, 0, 1, 0, 0, 0, 0, 0, 0];}{%
\[
\mathit{mot\_transmis} := [0, \,1, \,1, \,1, \,1, \,0, \,0, \,0, 
\,1, \,0, \,0, \,0, \,0, \,0, \,0]
\]
%
}
\end{maplelatex}

\begin{maplelatex}
\mapleinline{inert}{2d}{p_recu := X+X^2+X^3+X^4+X^8;}{%
\[
\mathit{p\_recu} := X + X^{2} + X^{3} + X^{4} + X^{8}
\]
%
}
\end{maplelatex}

\end{maplegroup}